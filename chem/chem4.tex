\documentclass[border=1pt]{standalone}
\usepackage[europeanresistors,americaninductors,americancurrents,siunitx]{circuitikz}
\usepackage{amssymb}


\begin{document}
    \begin{circuitikz}[]
        % \draw (0,0) coordinate(0)
        %     to [L] ++(3,0) coordinate(1);
        %     % to [L,l=4.35<\milli\henry>] ++(0,-3) coordinate(3);
        

        \draw (0,0)
            to ++(1,0) 
            to [L=$L_1/2$]++(2,0) coordinate(0)
            to [C=$2C_1$]++(2,0) coordinate(1)
            to ++(0,-0.5) coordinate(3)
            to ++(-0.5,0)
            to [L=$L_1$]++(0,-2)
            to ++(0.5,0) coordinate (2);
        \draw(3)
            to ++(0.5,0)
            to [C=$C_1$]++(0,-2)
            to ++(-0.5,0) coordinate (2)
            to ++(0,-0.5);
            

        \draw (1) 
            to [C=$2C_1$]++(2,0) coordinate(1)
            to [L=$L_1/2$]++(2,0) coordinate(1)
            to [C=$2C_1$]++(2,0) coordinate(1)
            to [L=$L_1/2$]++(2,0) coordinate(1)
            to ++(0,-0.5) coordinate(3)
            to ++(-0.5,0)
            to [L=$L_1$]++(0,-2)
            to ++(0.5,0) coordinate (2);
        \draw(3)
            to ++(0.5,0)
            to [C=$C_1$]++(0,-2)
            to ++(-0.5,0) coordinate (2)
            to ++(0,-0.5);

        \draw (1) 
            to [L>_=$L_1/2$]++(2,0) coordinate(1)
            to [C=$2C_1$]++(2,0) coordinate(1)
            to [R=$R$]++(0,-3)
            to (0,-3)[];


    \end{circuitikz}
\end{document}